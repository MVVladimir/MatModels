\documentclass[a4paper, 12pt]{article}

\usepackage[ a4paper,
 total={170mm,257mm},
 left=20mm,
 top=20mm,]{geometry}

\usepackage{cmap}
\usepackage[T2A]{fontenc}
\usepackage[utf8]{inputenc}
\usepackage[english, russian]{babel}

%Графика
\usepackage{graphicx}
\usepackage{float}%"Плавающие" картинки
\usepackage{wrapfig}%Обтекание фигур (таблиц, картинок и прочего)
\graphicspath{{./images/}}

% Математика
\usepackage{amsmath,amsfonts,amssymb,amsthm,mathtools} 

%Title Page
\title{ЛАБОРАТОРНАЯ РАБОТА № 1 \\
Моделирование механических систем
}
\author{Вариант 11 \\ Машуров Владимир БПМ-19-3}

\setcounter{page}{0}

\begin{document}
\maketitle
\thispagestyle{empty}
\newpage
\tableofcontents

\section{Моделирование механической системы масса-пружина}

Дана система:

\begin{equation}
M\dot{x} + B\dot{x} + kx = f(t)
\label{СистемаСБлокомНаПружинеИДемпфером}
\end{equation}

Где $f(t)$ - входное воздействие, $x(t)$ - выходное воздействие.
 
\textbf{Задание 1.1 } \\
Применив преобразование Лапласа (с нулевыми начальными условиями) найдите передаточную функцию модели: $ G(s) = \frac{X(s)}{F(s)} $ 

Найдём соотношение из которого получим G(t):


$$M\dot{x}(t) + B\dot{x}(t) + kx(t) = f(t) \; \; \; \frac{d}{dt} = \lambda $$

$$ M\lambda^2x(t) + B\lambda x(t) + kx(t) = f(t) $$

$$ (M\lambda^2 + B\lambda + k)x(t) = f(t) $$

$$ \frac{x(t)}{f(t)} = \frac{1}{M\lambda^2 + B\lambda + k} $$

Отсюда

$$ G(s) = \frac{X(s)}{F(s)} = \mathcal{L} \bigg( \frac{x(t)}{f(t)} \bigg) = \frac{\mathcal{L}(\tilde{x}(t))}{\mathcal{L}(\tilde{f}(t))} = $$
$$ = \frac{\mathcal{L}(1)}{\mathcal{L}(M\lambda^2 + B\lambda + k)} =  \frac{1}{s} \cdot \frac{s}{M\lambda^2 + B\lambda + k} = \frac{1}{M\lambda^2 + B\lambda + k} $$

\textbf{Задание 1.2 } \\
Перепишите уравнение \ref{СистемаСБлокомНаПружинеИДемпфером} в форму вход-состояние-выход.

$$M\ddot{x}(t) + B\dot{x}(t) + kx(t) = f(t) $$ 

$$ \begin{cases}
y_1 = x \\
y_2 = \dot{x} + kx - f
\end{cases} $$

Продифференцируем оба равенства по $t$

$$ \begin{cases}
\dot y_1 = \dot{x} \\
\dot y_2 = \ddot{x} + k \dot x - \dot f 
\end{cases} $$

Получим исходную систему

$$ \begin{cases}
\dot y_1 = y_2 - kx + f \\
\dot y_2 = f - ky_1 
\end{cases} $$

Мы пришли к форме вход-состояние-выход.

\textbf{Задание 1.3 } \\
Составьте структурную схему моделирования, опираясь на уравнение \ref{СистемаСБлокомНаПружинеИДемпфером} и результат, полученный в Задании 2.

$$M\dot{x}(t) + B\dot{x}(t) + kx(t) = f(t) \; \; \; \frac{d}{dt} = \lambda $$

$$ M\lambda^2x(t) + B\lambda x(t) + kx(t) = f(t) \; \; \; |:2 $$

$$ Mx + \frac{Bx}{\lambda} + \frac{kx}{\lambda^2} - \frac{f}{\lambda^2} = 0 $$

$$ \frac{1}{\lambda^2}(kx-f) + \frac{1}{\lambda}(Bx) + Mx = 0 $$

Из полученного выражения можно построить структурную схему \ref{p:Схема1}.

\begin{figure}[h!]
	\centering
	\includegraphics[scale=1]{scheme1}
	\caption{Структурная схема моделирования механической системы масса-пружина }
	\label{p:Схема1}
\end{figure}

\newpage
\section{Исследование модели вход-состояние-выход}

\end{document}